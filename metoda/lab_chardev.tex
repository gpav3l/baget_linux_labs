\chapter{Модуль ядра}\label{lab:char_dev}
\textbf{Цель:} Сборка модуля ядра, работа с device tree.

\vspace{5mm}
\textbf{Описание:}Для начала нужно понять, что в ОС Linux выделяется два пространства: пространство ядра (Kernel Space) и пользовательское пространство (User Space).

Эти два пространства разделены между собой, и взаимодействие может происходить только при помощи специального системного интерфейса (System Call Interface).

\vspace{5mm}
\textbf{Полезные ссылки:}
\begin{itemize}
	\item \href{https://linux-kernel-labs.github.io/refs/heads/master/labs/device_drivers.html}{Linux Kernel Labs: Character device drivers}.
	\item \href{https://www.digi.com/resources/examples-guides/use-device-tree-overlays-to-patch-your-device-tree}{DiGi: Use Device Tree Overlays to Patch Your Device Tree}.
\end{itemize}

В пространстве ядра работает, как не удивительно, само ядро ОС, а так же его модули (которые могут быть встроены в основное ядро, или могут быть представлены файлами в rootfs с расширением ko).

Пользовательские приложения и библиотеки работают в пользовательском пространстве. 

\begin{center}
	\includegraphics[width=0.6\textwidth]{pic_12}
\end{center}

Ядро ОС Linux умеет работать с подгружаемыми модулями (если эта опция была активирована при сборке ядра). В качестве отдельных модулей могут выступать как определённые сервисы, предоставляемые ядром, так и драйверы устройств.

В Linux выделяют три типа устройств:
\begin{itemize}
	\item символьное устройство — представлены в файловой системой, минимальный объём данных один символ (или байт), примеры: клавиатура, мышь, принтер, тачскрин, экран, камера, spi устройство, i2c устройство и т.д. 
	\item блочное устройство — представлено в файловой системе, минимальный объём это блок данных, работа ведётся путём монтирования устройства. Примеры: карты памяти, SD карты и пр.  
	\item пакетное устройство — не представлено в файловой системе, устройство для взаимодействия пакетами, примеры: сетевой интерфейс, Wi-Fi, Bluetooth и т.д. 
\end{itemize}

В данной работе, мы создадим шаблон для символьного устройства, с которым можно будет взаимодействовать через пользовательское приложение.

\section{Сборка модуля}

\subsection{}Запустите виртуальную машину. Логин и пароль для входа: student / usrstudent.

\subsection{}Подключите по USB плату к ПК. Проверьте, и при необходимости подключите USB устройство FTDI RBM\_C1K5500VK018 к виртуальной машине (меню Device→USB).

\subsection{}Откройте консоль комбинацией клавиш Ctrl+Alt+T

\subsection{}Добавьте ключи в ssh-agent
\begin{lstlisting}[style=bash]
# ssh-add
\end{lstlisting}
Впишите фразу заданую в пункте \ref{ssh:passphrase}

\subsection{}Создайте рабочий каталог
\begin{lstlisting}[style=bash]
# mkdir -p $BAGET/lab_03 
\end{lstlisting}

\subsection{}Скопируйте в него рабочие файлы
\begin{lstlisting}[style=bash]
# cp -r $BAGET/support/mychar $BAGET/lab_03/
\end{lstlisting}

\subsection{}Перейдите в рабочий каталог и запустите vscode
\begin{lstlisting}[style=bash]
# cd $BAGET/lab_03/mychar; code .
\end{lstlisting}

\begin{Notes}{Замечание:} 
	При первом входе, Вас могут спросить, доверяете ли вы автору, нужно нажать на кнопку Yes.
	\begin{center}
		\includegraphics[width=0.5\textwidth]{pic_13}
	\end{center}  
\end{Notes}

\subsection{}Добавим путей для разрешения части include директив. Откройте .vscode -> c\_cpp\_properties.json
\begin{center}
	\includegraphics[width=0.5\textwidth]{pic_14}
\end{center}
\begin{Notes}{Замечание:} 
	Если этого файла нет, то необходимо горячими клавишами Shift+Ctrl+P вызвать поле ввода комманд и выбрать C/C++: Edit Configuration (JSON) 
\end{Notes}


\subsection{}Допишите в поле includePath следующие строки:\\
\begin{center}
	\includegraphics[width=\textwidth]{pic_15}
\end{center}
для удобства, воспользуйтесь файлом vscode\_lines.md в каталоге support. Ctrl+S для сохранения

\subsection{}Откройте Makefile. Разберём основные моменты

obj-m  - тут мы задаём имя объектного файла верхнего уровня (файла из которого по итогу будет собран модуль). 

chardev-y — описываем из каких объектных файлов будет собираться итоговый. 

KERNEL\_SRC — тут мы создаём переменную, которая будет хранить путь к заголовочным файлам ядра.

SRC — в этой переменной мы записываем путь к исходникам модуля, через вызов утилиты pwd. Таким образом, можно рабочую папку переносить куда угодно, без ручных правок. При сборке модуля, текущий путь меняется на путь KERNEL\_SRC, поэтому путь к исходникам модуля ядра должен быть абослютным.

all: — список команд выполняемых при сборке 

clean: — список команд выполняемых при очистке

\subsection{}Изучим содержимое файла mychar\_init.c.

В данном файле содержится минимальный набор кода, необходимый для сборки модуля ядра, и для его функционирования в системе. Нужно помнить, что данный код выполняется в пространстве ядра (Kernel space), соответственно не все парадигмы и функции привычные по работе с приложениями тут будут работать, хотя большинство из них имеют тут аналоги (так к примеру вызов функции вывода текста через стандартный поток вывода - printf, приведёт к ошибке компиляции, и вместо неё нужно использовать её аналог - printk).

Обратите внимание на структура в районе 36 строки. Эта структура считывается ядром, при обращении к модулю, для установления связи между командами, отправляемыми через системный интерфейс, и функциями, которые их будут исполнять. Таким образом, достигается возможность исключить коллизий при работе множества различных драйверов. По той же причине рекомендуеться  все структуры и функции объявлять с ключевым словом static.

Следующий момент, это структура объявляемая со строки 25. Параметр .compatible используется ядром, при сканировании дерева устройств. И при наличии совпадения, автоматически загружает модуль ядра.  

\subsection{}Посмотрите на содержимое файла mychar\_dev.c. В этом файле описан процесс инициализации символьного устройства (функция int srisa\_pdrv\_chinit(struct platform\_device *pdev)), создание файла-заглушку в папке dev для доступа к конкретному аппаратному модулю (в данном случае, модуль ядра рассчитан на наличие только одного объекта, файл доступа будет /dev/srisa\_pdrv) а так же реализация API вызовов, адресованных данному устройству (открытие файла, закрытие файла, считывание и запись данных).
Обратите внимание на функции работы с данными (srisa\_pdrv\_read и srisa\_pdrv\_write), для обмена данными между user space и kernel space используются специальные функции copy\_to\_user и copy\_from\_user.

\subsection{}Если кликать мышкой на вызываемую функцию, при зажатой клавише Ctrl, то редактор откроет реализацию этой функции в отдельной вкладке (если реализация находиться в другом файле). Попробуйте, так как далее этот функционал понадобиться, при выполнении самостоятельных работ.

\subsection{}Перейдём в терминал (можно открыть вкладку TERMINAL в vscode, если её не видно, выберите в меню Terminal -> New Terminal) и соберём модуль командой
\begin{lstlisting}[style=bash]
# ARCH=mips CROSS_COMPILE=mips64el-linux-gnuabi64- make
\end{lstlisting}

\subsection{}В результате компиляции в консоли должен быть следующий вывод
\begin{center}
	\includegraphics[width=\textwidth]{pic_16}
\end{center}
В рабочей папке должны появится несколько файлов, в том числе chardev.ko

\section{Настройка платы}
Для работы с модулями в Linux есть ряд команд

\textbf{insmod <path\_to\_module>} - подключение модуля ядра, с явным указанием пути к файлу, при этом не проверяются зависимости (бывает что один модуль требует предварительного запуска другого модуля).

\textbf{modporbe <module\_name>} - это «умная» версия insmod, проверяет зависимости (файл modules.dep), и подгружает их при необходимости. Не требуется указывать полный путь к модулю, однако модуль должен находиться в папке  /lib/modules/<kernel\_version>/

\textbf{lsmod} — выводит список активированных модулей, с указанием счётчика зависимостей (показывает сколько потоков использует данный модуль)

\textbf{rmmod <module\_name>} - выгрузить указанный модуль (указывается только имя модуля, без расширения ko)

\subsection{}Откройте gtkterm и подключитесь к плате (/dev/ttyUSB1 скорость 115200), или откройте в теримнале новую вкладку, и вызовите утилиту minocom.

\subsection{}Для функционирования modporbe необходимо выполнить ряд подготовительных работ. В первую очередь создать каталог для хранения модулей и информации о них:
\begin{lstlisting}[style=bash]
$ mkdir -p /lib/modules/$(uname -r)/
\end{lstlisting}

\subsection{}Создадим два пустых файла
\begin{lstlisting}[style=bash]
$ touch  /lib/modules/$(uname -r)/modules.order
$ touch  /lib/modules/$(uname -r)/modules.builtin
\end{lstlisting}
В данном случае, допустим такой ход, но в общем случае, содержимое этих файлов должно отражать реальность, в частности обратите внимание, на наличие одноимённых файлов в рабочей папке, в которой находится исходный код модуля ядра. 
При этом пути к модулям, записанные в этих файлах ведуться от каталога, в котором они расположены.
Пример наполнения для modules.order и modules.builtin (не для текущей платы!):
\begin{lstlisting}[style=stdout]
kernel/kernel/configs.ko
kernel/fs/binfmt_script.ko
kernel/fs/mbcache.ko
\end{lstlisting}

\subsection{}Вернёмся в консоль виртуальной машины, и скопируем скомпилированный модуль на плату (проверьте, что плата подключена сетевым шнуром к рабочему ПК)
\begin{lstlisting}[style=bash]
# scp $BAGET/lab_03/mychar/chardev.ko \
netuser@192.168.100.200:/home/netuser/
\end{lstlisting}


\subsection{}Вернитесь в терминал платы, и переместите модуль ядра в системную папку
\begin{lstlisting}[style=bash]
$ mv /home/netuser/chardev.ko /lib/modules/$(uname -r)/
\end{lstlisting}

\subsection{}Определим зависимости для модуля, для чего выполните команду 
\begin{lstlisting}[style=bash]
$ depmod
\end{lstlisting}
Данная команда просканирует все модули, в папке /lib/modules/\$(uname -r)/ и запишет о них информацию. После чего модули могут подгружатся командой modprobe. 

\subsection{}Проверим работу нашего модуля, выполнив команду
\begin{lstlisting}[style=bash]
$ modprobe chardev
\end{lstlisting}
в результате, в консоли должны увидеть надпись 
\begin{lstlisting}[style=stdout]
srisa_pdrv: srisa_pdrv_init
\end{lstlisting}

\subsection{}Просмотрим список загруженных модулей
\begin{lstlisting}[style=bash]
$ lsmod
\end{lstlisting}
вывод должен быть следующим:
\begin{lstlisting}[style=stdout]
Module                  Size  Used by
chardev                 4620  0
\end{lstlisting}
Size - показывет сколько оперативной памяти выделено для выбранного модуля, Used by - сколько программ или модулей используют на данный момент указаный модуль (очевидно показываются только активные приложения и модули). 

\subsection{}Выполните команду ls /dev/* Обратите внимание, что отсутствует файл srisa\_pdrv, так как он создаётся, только при вызове функции pdrv\_probe, вызываемая только при наличии совместимого периферийного модуля.

\subsection{}Выгрузим из памяти наше ядро
\begin{lstlisting}[style=bash]
$ rmmod chardev
\end{lstlisting}

\section{Наложенное дерево устройств}
Мы успешно собрали модуль устройства, однако его сейчас никто не использует. Как говорилось выше, часто модули ядра являются драйвером, для взаимодействия с периферией (как внутрипроцессорной, типа интерфейса spi, так и внешней). Для описания взаимодействия между ядром я устройством служит специальная структура - дерево устройств (device tree). В ней описывается основная информация о каждом устройстве, подключенным к процессорному ядру, а так же могут добавлять параметры, применяемые модулем ядра, при работе с тем или иным устройством. Среди прочего, дерево устройств поддерживает функцию наложения (overlay), что позволяет создавать файлы, вносящие исправление в существующее описание, или дополняющее его, в соответствии с требованиями конкретного проекта, без необходимсоти пересобирать всё дерево устройств целиком.

\subsection{}Создадим рабочую папку в виртуальной машине для реализации наложеного дерева устройств и перейдём в неё
\begin{lstlisting}[style=bash]
# mkdir $BAGET/lab_03/devtree
# cd $BAGET/lab_03/devtree 
\end{lstlisting}

\subsection{}Создадим файл и откроем его для редактирования
\begin{lstlisting}[style=bash]
# kate of_mychard.dts
\end{lstlisting}

\subsection{}Запишите в файл следующие строки
\begin{lstlisting}[style=stdout]
/dts-v1/;

/ {
	fragment@0 {
		target-path = "/";
		__overlay__ {
			
			pdrv1 {
				compatible = "srisa,platform-device";
				status = "okay";
			};
		};
	};
};
\end{lstlisting}

\subsection{}Сохраните Ctrl+S и закройте редактор Ctrl+Q

На данном шаге было создано дерево устройств, которое описывает новое виртуальное устройство, совместимое с драйвером, созданом ранее (обратите внимание на поле compatible в написанном фрагменте, и сравните его значение со значением одноименного параметра в исходном коде модуля). Ключевое слово overlay показыввет, что данное дерево устройств должно быть наложено на уже существующее.  

\subsection{}Скомпилируем наложеное дерево устройств, для этого выполним следующую команду 
\begin{lstlisting}[style=bash]
# dtc -@ -O dtb -o ./of_mychard.dtb ./of_mychard.dts
\end{lstlisting}

\begin{Notes}{Запомните:}
	Компиляция наложенного дерева устройств от обычного не отличается. 
\end{Notes}

\vspace{5mm}
Особенность дерева устройств в том, что скомпилированное дерево устройств, можно вернуть в «человеческий вид» командой 
\begin{lstlisting}[style=bash]
# dtc -@ -O dts -o ./out.dts ./source.dtb
\end{lstlisting}

\subsection{}Скопируйте полученный файл на плату
\begin{lstlisting}[style=bash]
# scp $BAGET/lab_03/devtree/of_mychard.dtb \
netuser@192.168.100.200:/home/netuser/
\end{lstlisting}

\subsection{}Перейдите в терминал платы, и переместите скопированный файл в папку barebox
\begin{lstlisting}[style=bash]
$ mv /home/netuser/of_mychard.dtb /barebox/
\end{lstlisting}

\subsection{}Отредактируйте скрипт barebox.sh
\begin{lstlisting}[style=bash]
$ nano /barebox/barebox.sh
\end{lstlisting}
и впишите в него, после строки DTB=k5500vk018\_rbm.dtb
\begin{lstlisting}[style=stdout]
fdt_apply -i $DTB -l of_mychard.dtb -o /dtb && DTB=/dtb
\end{lstlisting}
Ctrl+S, Ctrl+X 

\subsection{}Перезагрузите плату командой reboot

\section{Проверка работы модуля}

\subsection{}Войдите как пользователь root, и выполните команду lsmod.
Модуль должен быть уже загружен, так как ядро при запуске должно было обнаружить в дереве устройств, совместимое с нашим модулем устройство. 

\subsection{}Посмотрим последние строки лога загрузки ядра
\begin{lstlisting}[style=bash]
$ dmesg | tail
\end{lstlisting}
В консоли должны быть следующие строки (значение в квадратных скобках могут отличаться, это значение секунд прошедших с момента запуска ядра операционной системы)
\begin{lstlisting}[style=stdout]
...
[ 27.263923] srisa_pdrv: srisa_pdrv_init
[ 27.266836] srisa_pdrv: pdrv_probe
...
\end{lstlisting}

Строку с init получали ранее, но теперь после неё есть строка srisa-pdrv: pdrv\_probe, которая показывает, что  модуль успешно связался с устройством описанным в наложенном дереве устройств.

\subsection{}Выполните команду ls /dev/* и обратите внимание, что должен появиться файл srisa\_pdrv в выдаче.

\subsection{}Отправим данные в srisa\_pdrv, и проанализируем сообщения в консоли
\begin{lstlisting}[style=bash]
$ echo 1234 > /dev/srisa_pdrv
\end{lstlisting}
\begin{lstlisting}[style=stdout]
srisa_pdrv: srisa_pdrv_open
srisa_pdrv: srisa_pdrv_ioctl
srisa_pdrv: srisa_pdrv_write: Get 5 bytes
srisa_pdrv: srisa_pdrv_release
\end{lstlisting} 

\vspace{3mm}
\begin{Notes}{Замечание:} 
	Принято пять байт, так как последний байт это байт перевода каретки (\textbackslash{}r). 
\end{Notes}
	
\subsection{}Посмотрим на скорость записи данных выполнив команду:
\begin{lstlisting}[style=bash]
$ dd count=1 bs=1M if=/dev/zero of=/dev/srisa_pdrv 
\end{lstlisting}
Команда dd может часто встречаться при активной работе с Linux. Она обеспечивает считывание, или копирование данных из одного файла в другой. В данном случае, мы указываем, что ходим один раз (count=1) отправить 1 мегабайт (bs=1M)  из файла /dev/zero (генератор 0) в драйвер. Чем больше отправим данных, тем точнее получим значение скорости, однако из-за особенностей реализации считывания в драйвере, большой объём данных может привести к печальным последствиям, так что увлекатся не стоит. Результат должен быть таким
\begin{lstlisting}[style=stdout]
srisa_pdrv: srisa_pdrv_open
srisa_pdrv: srisa_pdrv_write: Get 1048576 bytes
srisa_pdrv: srisa_pdrv_release 
1+0 records in
1+0 records out
1048576 bytes (1.0 MB, 1.0 MiB) copied, 0.0318856 s, 32.9 MB/s
\end{lstlisting}
Последняя строка показывает сколько данных было передано, и на какой скорости. Результат может менятся от каждого запуска, но порядок величин должен быть одинаковый.

\subsection{}Попробуем считать данные из драйвера, для этого выполним следующую команду: 
\begin{lstlisting}[style=bash]
$ dd count=1 bs=10 if=/dev/srisa_pdrv
\end{lstlisting}
Тут опять используеться команда dd, но на этот раз не указывается выходной файл (аргумент of), поэтому появляется возмоэность увидеть вывод драйвера в консоли
\begin{lstlisting}[style=stdout]
srisa_pdrv: srisa_pdrv_open 
srisa_pdrv: srisa_pdrv_read: Send 10 bytes 
ABCDEFGHIsrisa_pdrv: srisa_pdrv_release 
1+0 records in 
1+0 records out 
10 bytes copied, 0.0131135 s, 0.8 kB/s
\end{lstlisting}
Как видите, в выводе присутствует набор букв, который был сгенерирован драйвером.  

\subsection{}Оценим скорость чтения данных, но на этот раз перенаправим вывод в файл, в противном случае придётся долго ждать завершения вывода в консоль запрошенного объёма данных (или прервать его комбинацией клавиш Ctrl+C): 
\begin{lstlisting}[style=bash]
$ dd count=1 bs=1M if=/dev/srisa_pdrv of=/root/drv.out
\end{lstlisting}
\begin{lstlisting}[style=stdout]
srisa_pdrv: srisa_pdrv_open 
srisa_pdrv: srisa_pdrv_read: Send 1048576 bytes 
srisa_pdrv: srisa_pdrv_release 
1+0 records in 
1+0 records out 
1048576 bytes (1.0 MB, 1.0 MiB) copied, 0.140926 s, 7.4 MB/s
\end{lstlisting}
Скорость на чтение меньше, чем на запись, так как драйвер в начале заполняет свой внутренний массив, и затем отправляет данные. Так же тормозит и запись файла. Проверьте это, заменив файл назначения на /dev/null Для просмотра вывода выполните команду cat /root/drv.out и немного подождём, пока через терминал будет передан 1 мегабайт данных. Вывод можно прервать комбинацией Ctrl+C.

\subsection{}Выгрузим ядро из памяти
\begin{lstlisting}[style=bash]
$ rmmod chardev
\end{lstlisting}
Отметим, что есть два вывода, один от модуля, который получали и ранее, второй сообщает, что была выполнена функция по отключению от устройства (удалению файла заглушки из папки /dev).  

\subsection{} Выключите плату, для чего в начале введите команду
\begin{lstlisting}[style=bash]
	$ poweroff
\end{lstlisting}
дождитесь, как появится надпись
\begin{lstlisting}[style=stdout]
	reboot: System halt
\end{lstlisting}
после чего отключите USB кабель от ПК или платы.  

\section{Задание для самостоятельной работы}
Внесите правки в модуль ядра, так, что бы командой считывания можно было получить все данные, записанные ранее через команду записи. Другими словами, пользователь два раза записал Hello после чего запросил на считывание 1Мбайт, в ответ он должен получить 10 байт: HelloHello. При этом символ переноса картеки (\textbackslash{}r) должен быть исключён.

Тут понадобится работа со структурой kfifo (в коде переменная data\_fifo). В коде уже вставлены части по её инициализации (mychar\_dev.c), Вам остаётся вписать в нужные места вызовы функций для считывания и записи данных. При этом функция copy\_from\_user и copy\_to\_user уже не нужны.

Командой dd можно передавать данные из любых файлов, что может помочь в проверке работы драйвера. Как один из вариантов, создайте текстовый файл, и впишите в него текст для передачи в драйвер. Через комманду dd передаёте этот файл в драйвер, и второй командой dd считываете данные в другой файл, после чего можете сравнить два файла (исходный и полученый) командой diff <path1> <path2>

Пример для записи данных в драйвере с применение структуры kfifo.
\begin{lstlisting}[style=stdout]
static ssize_t fifo_write(struct file *file, const char __user *buf, size_t count, loff_t *ppos)
{
	int ret;
	unsigned int copied;
	
	ret = kfifo_from_user(&test, buf, count, &copied);
	
	return ret ? ret : copied;
}
\end{lstlisting}


Пример для считывания данных из драйвера с использованием структуры kfifo.
\begin{lstlisting}[style=stdout]
static ssize_t fifo_read(struct file *file, char __user *buf, size_t count, loff_t *ppos)
{
	int ret;
	unsigned int copied;
	
	ret = kfifo_to_user(&test, buf, count, &copied);
	
	return ret ? ret : copied;
}
\end{lstlisting}

\subsubsection{*} Модифицировать код таким образом, что бы  можно было независимо обрабатывать два устройства (для проверки добавьте в device tree ещё одно устройство, pdrv2)
