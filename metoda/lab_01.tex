\chapter{Установка ОС на целевую платформу}
\textbf{Цель:} Установить дистрибутив Debian 10 на целевую платформу.

\textbf{Описание:} Нужно понимать, что ОС Linux делиться на две части. Ядро (Kernel) и файловую систему (rootfs). Ядро определяет какие возможности по взаимодействию с железом доступны пользователю и приложениям. Rootfs определяет какие инструменты есть в системе, порядок инициализации, запуска сервисов и прочие прикладные штуки. Дистрибутив это в большей степени наполнение rootfs. Не каждый дистрибутив имеет поддержку иной от x86 архитектур.

Для установки debian воспользуемся инструментом под названием debootstrap, который отвечает за создание rootfs. Подробнее про этот инструмент, и возможности по его управлению, Вы можете найти на wiki проекта Debian.
\section{Подготовка rootfs}

\subsection{}Запустите виртуальную машину. Логин и пароль для входа: student / usrstudent.

\subsection{} Запустите консоль нажав комбинацию клавиш на клавиатуре \textbf{Ctrl+Alt+T} \\

\textbf{Внимание!} В консоли есть возможность автоматического продолжения ввода. Для этого необходимо ввести первые символы команды, или пути, и нажать клавишу TAB. Ввод продолжиться до тех пор, пока не появиться неопределённостью (к примеру у Вас есть два файла foo\_bar1 и foo\_bar2, при нажатии TAB будет вставлен текст до цифры). Двойное нажатие TAB приведёт к выводу всех возможных вариантов продолжения, если таковые есть.

\subsection{}Создайте и перейдти в рабочий каталог в котором будет создан образ rootfs.
\begin{lstlisting}
# mkdir -p $BAGET/lab_01
# cd $BAGET/lab_01 
\end{lstlisting}

\subsection{} Запустите утилиту debootstrap (при необходимсоти введите пароль usrstudent):
\begin{lstlisting}
# sudo debootstrap --include=aptitude,nano,wget \
--foreign \
--arch=mips64el buster rootfs
\end{lstlisting}
\textit{-{}-include=A,B,C..} - добавить в сборку указанные пакеты \\
\textit{-{}-foreign} — только сгенерировать наполнение, применяется когда архитектура на которой запускается утилита отлична от архитектуры назначения. \\
\textit{-{}-arch=mips64el} — указываем целевую архитектуру \\
\textit{buster} — версия сборки \\
\textit{rootfs} — путь к папке назначения, где будут размещены файлы \\

\subsection{} Для продолжения установки, нам понадобиться утилита qemu позволяющая эмулировать различные архитектуры. Скопируем исполняемый файл qemu:
\begin{lstlisting}
# sudo cp /usr/bin/qemu-mips64el-static ./rootfs/usr/bin
\end{lstlisting}
и перейдём в созданную rootfs (привет дедушка контейнеров, chroot)
\begin{lstlisting}
# sudo chroot ./rootfs
\end{lstlisting}

\section{Настройка rootfs}

\subsection{} Завершим работу debootstrap
\begin{lstlisting}
# export LANG=en_US.UTF-8
# /debootstrap/debootstrap --second-stage
\end{lstlisting}

\subsection{}Добавим источники для установки ПО, для этого выполним следующие команды 
\begin{lstlisting}
# echo deb http://ftp.debian.org/debian buster \
main contrib non-free  >> /etc/apt/sources.list

# echo deb-src http://ftp.debian.org/debian buster \
main contrib non-free >> /etc/apt/sources.list

# echo deb http://ftp.debian.org/debian buster-updates \
main contrib non-free >> /etc/apt/sources.list

# echo deb-src http://ftp.debian.org/debian buster-updates \
main contrib non-free >> /etc/apt/sources.list
\end{lstlisting}

\subsection{} Обновим список, и установим ряд приложений
\begin{lstlisting}
# apt-get update
# apt-get install -y dialog sudo less i2c-tools evtest mc \
openssh-server resolvconf hwinfo net-tools
\end{lstlisting}

\subsection{}Зададим пароль для root пользователя
\begin{lstlisting}
# passwd root
\end{lstlisting}
После чего введите root (внимание, курсор двигаться не будет, это политика безопасности Linux, при вводе пароля курсор не перемещается, что бы нельзя было установить количество символов). И нажмите Enter

Затем Вас попросят повторить пароль, снова введите root и нажмите Enter